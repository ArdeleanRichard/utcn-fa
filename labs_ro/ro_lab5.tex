\documentclass[../ro-fa-lab.tex]{subfiles}
\usepackage{booktabs}
\usepackage{tabularx}
\usepackage{caption}


\usepackage{hyperref}
\hypersetup{
    pdftitle={(RO) L5 - Tabele de dispersie},   % The title shown in the browser tab
    pdfauthor={},         % Your name or organization
    pdfsubject={},   % A brief description
    pdfkeywords={}
}

\begin{document}


\section{\texorpdfstring{\textbf{Tema Nr. 5: Căutarea în tabele de dispersie}}{Tema Nr. 5: Căutarea în tabele de dispersie}}\label{assign5}

\begin{quote}

\textbf{Adresare deschisă, verificare pătratică}
\end{quote}

\textbf{Timp alocat:} 2 ore

\subsection{Implementare}\label{implementare}

Se cere implementarea \textbf{corectă} și \textbf{eficientă} a
operațiilor de \emph{inserare} și \emph{căutare} într-o tabelă de
dispersie ce folosește \emph{adresarea deschisă} cu \emph{verificare
pătratică}.

Informații utile și pseudo-cod găsiți în notițele de curs sau în carte
(Cormen), in secțiunea \emph{11.4 Open} \emph{addressing}.

Noțiunile închis/deschis (closed/open) specifică dacă se obligă
folosirea unei poziții sau structuri de date.

\subsubsection{Hashing (se referă la tabela de dispersie (hash
table))}\label{hashing-se-referux103-la-tabela-de-dispersie-hash-table}

\begin{itemize}
\item
  \textsubscript{Open Hashing}

  \begin{itemize}
  \item
    Pe o anumită poziție se pot stoca mai multe elemente
    (ex: chaining)
  \end{itemize}
\item
  Closed Hashing

  \begin{itemize}
  \item
    Se poate stoca doar un singur element pe o anumita
    pozitie (ex. linear/quadratic probing)
  \end{itemize}
\end{itemize}

\textbf{Addressing (se referă la poziția finală a unui element față de
poziția inițială)}

\begin{itemize}
\item
  Open Addressing (Adresare Deschisă)

  \begin{itemize}
  \item
    Adresa finală (poziția finală) nu este complet
    determinată de către codul hash. Poziția depinde și de elementele
    care sunt deja în tabelă. (ex: linear/quadratic probing - verificare
    liniară/pătratică)
  \end{itemize}
\item
  \textsubscript{Closed Addressing (Adresare Închisă)}

  \begin{itemize}
  \item
    Adresa finală este întotdeauna determinată the codul
    hash (poziția inițială calculată) și nu exista probing (ex:
    chaining)
  \end{itemize}
\end{itemize}

Pentru această temă tabela de dispersie va avea o structură similară cu
cea de mai jos\textsubscript{:}

\emph{typedef struct \{ int id; char name{[}30{]}; \} Entry;}

unde \emph{adresa finală} în tabelă va fi calculată aplicând funcția de
hashing pe câmpul \emph{id} din structură. Câmpul \emph{name} va fi
folosit doar pentru demonstrarea corectitudinii operațiilor de căutare
și ștergere și nu este necesară pentru evaluarea performanței (câmpul
\emph{name} va fi afișat în consolă dacă operația de căutare găsește
\emph{id}-ul respectiv, altfel afișați ``negăsit'').

\subsection{Cerințe minimale pentru
notare}\label{cerinux21be-minimale-pentru-notare}

\begin{itemize}
\item
  Interpretați graficul și notați observațiile personale în antetul
  fișierului \emph{main.cpp}, într-un comentariu bloc informativ.
\item
  Pregătiți un exemplu pentru exemplificarea corectitudinii fiecărui
  algoritm implementat.
\item
  Nu preluăm teme care nu sunt indentate și care nu sunt organizate în
  funcții (de exemplu nu preluam teme unde tot codul este pus in main)
\item
  \emph{\textbf{Punctajele din barem se dau pentru rezolvarea corectă și
  completă a cerinței, calitatea interpretărilor din comentariul bloc și
  răspunsul corect dat de voi la întrebările puse de către profesor.}}
\end{itemize}

\subsection{Cerințe}\label{cerinux21be}

\subsubsection{\texorpdfstring{Implementarea operației de inserare și
căutare folosind structura de date cerută + \emph{demo (dimensiune 10)}
(5p)}{Implementarea operației de inserare și căutare folosind structura de date cerută + demo (dimensiune 10) (5p)}}\label{implementarea-operaux21biei-de-inserare-ux219i-cux103utare-folosind-structura-de-date-cerutux103-demo-dimensiune-10-5p}

Corectitudinea algoritmilor va trebui demonstrată pe date de intrare de
dimensiuni mici (ex. 10).

\subsubsection{Evaluarea operației de căutare pentru un singur factor de
umplere 95\%
(2p)}\label{evaluarea-operaux21biei-de-cux103utare-pentru-un-singur-factor-de-umplere-95-2p}

\textbf{!} Înainte de a începe să lucrați pe partea de evaluare,
asigurați-vă că aveți un \textbf{algoritm corect}!

Se cere evaluarea operației de \emph{căutare} în tabele de dispersie cu
adresare deschisă și verificare pătratică, în cazul \textbf{mediu
statistic} (nu uitați să repetați măsurătorile de 5 ori). Pentru a
obține evaluarea, trebuie să:

\begin{enumerate}
\def\labelenumi{\arabic{enumi}.}
\item
  Alegeți \emph{N}, dimensiunea tabelei, un număr prim în jur de 10000
  (e.g. 9973, sau 10007);
\item
  Pentru fiecare din următoarele valori pentru factorul de umplere
  \textbf{$\alpha$} \textbf{= 0.95}:

  \begin{enumerate}
  \def\labelenumii{\alph{enumii}.}
  \item
    Inserati în tabela \emph{n} elemente aleator, astfel incat sa
    ajungeți la valoare lui \emph{$\alpha$} (\emph{$\alpha$ =} \emph{n/N})
  \item
    Căutați aleator, în fiecare caz, \emph{m} elemente (\emph{m}
    \textasciitilde{} 3000), astfel încât aproximativ jumătate din
    elemente sa fie \emph{găsite}, iar restul sa \emph{nu fie găsite}
    (în tabelă). \emph{Asigurați-vă că elementele găsite sunt generate
    uniform, i.e. să căutați elemente care au fost introduse la moment
    diferite, cu probabilitate egală (există mai multe moduri în care se
    poate garanta acest lucru)}
  \item
    Numărați operațiile efectuate de procedura de căutare (i.e. numărul
    de celule accesate)
  \item
    Atenție la valorile pe care le căutați, să fie într-o ordine
    aleatoare din cele introduse. \emph{Dacă sunt primele 1500 adăugate,
    implicit efortul mediu găsite va fi 1.}
  \end{enumerate}
\item
  Generați un tabel de forma:
\end{enumerate}

\begin{table}[ht]
  \centering
  \caption{Effort measurements at various filling factors}
  \label{tbl:effort-vs-filling}
  \begin{tabularx}{\textwidth}{%
      >{\centering\arraybackslash}X
      >{\centering\arraybackslash}X
      >{\centering\arraybackslash}X
      >{\centering\arraybackslash}X
      >{\centering\arraybackslash}X
    }
    \toprule
    \textbf{Filling factor}
      & \textbf{Avg. Effort \emph{(found)}}
      & \textbf{Max Effort \emph{(found)}}
      & \textbf{Avg. Effort \emph{(not-found)}}
      & \textbf{Max Effort \emph{(not-found)}} \\
    \midrule
    0.95 & … & … & … & … \\
    % add more rows here as needed
    \bottomrule
  \end{tabularx}
\end{table}

\emph{Efort mediu = efort\_total / nr\_elemente}

\emph{Efort maxim = număr maxim de accese efectuat de o operație de
căutare}

\subsubsection{Completarea evaluării pentru toți factorii de umplere
(2p)}\label{completarea-evaluux103rii-pentru-toux21bi-factorii-de-umplere-2p}

Respectând cerințele de la punctul 2 cu \emph{$\alpha$} $\in$ \{0.8, 0.85, 0.9,
0.95, 0.99\}, generați un tabel de forma:

\begin{table}[ht]
  \centering
  \caption{Effort measurements at various filling factors}
  \label{tbl:effort-vs-filling-extended}
  \begin{tabularx}{\textwidth}{%
      >{\centering\arraybackslash}X
      >{\centering\arraybackslash}X
      >{\centering\arraybackslash}X
      >{\centering\arraybackslash}X
      >{\centering\arraybackslash}X
    }
    \toprule
    \textbf{Filling factor}
      & \textbf{Avg. Effort \emph{(found)}}
      & \textbf{Max Effort \emph{(found)}}
      & \textbf{Avg. Effort \emph{(not-found)}}
      & \textbf{Max Effort \emph{(not-found)}} \\
    \midrule
    0.80  &     &     &     &     \\
    0.85  &     &     &     &     \\
    …     & …   &     & …   &     \\
    \bottomrule
  \end{tabularx}
\end{table}

\subsubsection{\texorpdfstring{Implementare operației de ștergere într-o
tabelă de dispersie, \emph{demo (dimensiune 10)} și evaluarea operației
de căutare după ștergerea unor elemente
(1p)}{Implementare operației de ștergere într-o tabelă de dispersie, demo (dimensiune 10) și evaluarea operației de căutare după ștergerea unor elemente (1p)}}\label{implementare-operaux21biei-de-ux219tergere-uxeentr-o-tabelux103-de-dispersie-demo-dimensiune-10-ux219i-evaluarea-operaux21biei-de-cux103utare-dupux103-ux219tergerea-unor-elemente-1p}

Pentru evaluarea operației de de căutare după ștergere, umpleți tabela
de dispersie pâna la factor de umplere 0.99. Stergeti elemente din
tabela până ajungeți la factor umplere 0.8, după care căutați m elemente
aleator (m \textasciitilde{} 3000), astfel incat aproximativ jumătate
din elemente sa fie \emph{găsite}, iar restul sa nu fie găsite (în
tabela). Măsurați efortul necesar pentru \emph{căutare} și adăugați în
tabelul generat anterior.

\end{document}