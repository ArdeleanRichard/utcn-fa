\documentclass[../en-fa-lab.tex]{subfiles}

\usepackage{hyperref}
\hypersetup{
    pdftitle={Ghid de laborator},   % The title shown in the browser tab
    pdfauthor={},         % Your name or organization
    pdfsubject={},   % A brief description
    pdfkeywords={}
}


\begin{document}

\section{\texorpdfstring{\textbf{Ghid de laborator}}{Ghid de laborator}}\label{ghid}

\subsection{Despre laborator}\label{despre-laborator}

În acest semestru, urmează să implementați o serie de algoritmi și să
analizați corectitudinea și eficiența propriilor implementări.
Implementările vor avea ca punct de pornire pseudo-codul de la curs și
seminar. Le puteți scrie în C sau C++ sau un alt limbaj de programare pe
care îl cunoașteți atâta timp cât implementați toate structurile de date
de care aveți nevoie.

\subsection{Formatul laboratorului}\label{formatul-laboratorului}

\subsubsection{Reguli}\label{reguli}

\begin{itemize}
\item
  Prezența este obligatorie
\item
  O absență poate fi recuperată (tema corespunzătoare poate fi predată
  în următoarea săptămână)
\item
  Absența a 2-a poate fi recuperata în laboratorul special de la
  sfârșitul semestrului
  (\href{http://ac.utcluj.ro/regulamente.html?file=files/regulamente_studenti/Taxe_2015-2016.pdf}{\ul{contra cost}}) rezolvând teme suplimentare (NU se rezolva tema
  corespunzatoare sesiunii in care s-a inregistrat absenta)
\item
  Dacă ai mai mult de 2 absențe \textbf{nu poţi participa la examen în
  sesiunea normală}
\item
  \textbf{IN MOD EXCEPTIONAL}, poți participa la laborator în aceeași
  săptămână cu o altă grupă dacă mă anunți din timp pe email, si
  primesti acordul meu (a cadrului didactic de laborator de la care
  lipsesti); tema se prezinta tot cadrului didactic la sesiunile căruia
  ești înscris (tema se incarcă la timp, se prezintă data urmatoare când
  participi cu semigrupa ta)
\end{itemize}

\subsubsection{Notare}\label{notare}

\begin{itemize}
\item
  Nota de la laborator valorează \textbf{30\%} din nota finală
\item 
    Nota de la laborator este compusă din două părți: Nota pe teme - 2/3 din nota de laborator (altfel spus \textbf{20\%} din nota finală) și Colocviu - 1/3 din nota de laborator (altfel spus \textbf{10\%} din nota finală)
\item Teme
\begin{itemize}
    \item
        Fiecare temă are o pondere egală la calcului notei finale pe teme - media aritmetică
    \item
        Fiecare temă are mai multe praguri de notare accesibile în documentul de cerințe
\end{itemize}

\item Colocviu
\begin{itemize}
    \item 
        Colocviul constă într-un test de laborator de tip closed book în ultima săptămână a semestrului (W14)
    \item 
        Colocviul va fi susținut de fiecare student pe calculatoarele disponibile în sala de laborator (nu pe laptopuri / calculatoare personale)
\end{itemize}
\item Pentru a promova laboratorul și a fi acceptați în examen, trebuie să îndepliniți \textbf{ambele cerințe}:
\begin{itemize}
    \item Nota teme $\geq 5$
    \item Nota colocviu $\geq\ 5$
\end{itemize}

% \item
%   Accederea în examen e condiționată de o \textbf{medie de minim 5 la
%   laborator}
\end{itemize}

%\subsubsection{Teme}\label{teme}

\subsubsection{Predare teme}\label{predare}

\begin{itemize}
\item
  La discuția de evaluare se vor prezenta: \textbf{codul sursă,
  graficele} și un \textbf{exemplu de rulare}
\item
  \textbf{Codul sursă} (ex: program.cpp) si \textbf{graficele} trebuie
  încărcate pe Moodle, într-o arhivă (ex: program.zip), \textbf{
  înainte de sesiunea de laborator}
\item
  \textbf{Nu evaluăm} teme cu cod neindentat
\item
  \textbf{Nu evaluăm} teme pentru care studentul nu poate explica
  algoritmul (algoritmii) utilizați
\item
  Fiecare fișier sursă trebuie să conțină la început un comentariu cu
  următorul format:
\end{itemize}

/**

* @author Ionescu Popescu

* @group 30221

*

* Specificațiile problemei, ex: Comparați metodele de sortare X, Y

*

* Interpretarea personală despre complexitate (timp și spațiu), despre
cazurile de testare (favorabil,

* mediu-statistic si nefavorabil) ex: Metoda X are complexitatea Y in
cazul Z pentru ca ..

*/

\subsubsection{Evaluarea complexității
algoritmilor}\label{evaluarea-complexitux103ux21bii-algoritmilor}

\begin{itemize}
\item
  Pentru cazul mediu-statistic, repetați măsurătorile de cel puțin 5 ori
\item
  Măsurați numărul de operații efectuate de algoritm (atribuiri și
  comparații pe datele de intrare sau pe variabile auxiliare ce conțin
  date de intrare)
\item
  Variați dimensiunea datelor de intrare în concordanță cu specificația
  fiecărei teme
\item
  Aplicați aceleași date de intrare pe fiecare algoritm în cazul
  evaluărilor comparative (cazul mediu statistic)
\item
  Generați grafice pentru evaluare (fie în \textbf{Excel} sau folosind
  \textbf{Profiler}-ul)
\item
  Analizați graficele și adaugați observațiile personale în secțiunea de
  început
\end{itemize}

\subsubsection{Termene de predare}\label{termene-de-predare}

Temele pot fi predate:

\begin{itemize}
\item
  În cadrul laboratorului în care sunt discutate. La finalul laboratorului trebuie sa încărcați pe \textbf{moodle} o versiune a temei curente (cu cat ați apucat să lucrați la ea). Lipsa unei submisii la finalul orei (sau a unei submisii cu prea puțin cod relevant) se penalizează cu pana la 2 puncte din nota pe acea tema.
\item
  \textbf{Extensia\_1} (E1): la începutul următorului laborator
\item
  \textbf{Extensia\_2} (E2): \ul{anumite} teme se pot preda la începutul
  celui de-al doilea laborator de după cel în care a fost prezentata
  tema (cu o \textbf{penalizare} de -2)
\item
  Din cel de-al treilea laborator, tema nu mai poate fi predata
  (valorează 0)
\end{itemize}

Găsiți o planificare a temelor și a extensiilor corespunzătoare pe
Moodle.

Temele trebuie încărcate pe Moodle înainte de începutul laboratorului în care sunt predate.

\subsubsection{Tentativa de fraudare}\label{tentativa-de-fraudare}

Pentru prima tentativă de fraudare descoperită (copiatul codului altei persoane sau folosirea de cod generat prin unelte AI), tema respectivă se punctează cu 0 puncte. \emph{\textbf{O tentativă ulterioară de fraudare duce la recontractarea materiei anul următor}}.

\subsection{Transferul între cadre
didactice}\label{transferul-uxeentre-cadre-didactice}

Dacă doriți să participați la orele de laborator cu un alt cadru
didactic trebuie să respectați următoarea regulă:

\begin{itemize}
\item
  Studentul S1 din grupa G1 poate să se mute in grupa G2 doar dacă găsește un student S2 din grupa G2 ce e dispus să participe la laborator cu grupa G1.
\end{itemize}

Pentru a formaliza "transferul" trebuie să trimiteți un email în care să
precizați cu cine faceți transferul:

\begin{itemize}
\item
  Un email de la S1 către cele două cadre didactice implicate
\item
  Un email de la S2 către cele două cadre didactice implicate
\end{itemize}

Termenul limita pentru "transfer" e sfârșitul \textbf{săptămânii 2} de
scoala.

\subsection{Bibliografie sugerată}\label{bibliografie-sugeratux103}

\begin{itemize}
\item
    \href{https://drive.google.com/drive/folders/12NX3D8DLrPuU-9ffI9MjnoKUNjjftWRg?usp=sharing}{Cormen, T. H. et al (2009). Introduction to algorithms. MIT press}
  
\item
    \href{https://drive.google.com/drive/folders/12NX3D8DLrPuU-9ffI9MjnoKUNjjftWRg?usp=sharing}{J. Kleinberg, E. Tardos (2005). Algorithm Design. Addison Wesley}
\item
  Tutoriale C/C++

  \begin{itemize}
  \item
    \href{http://www.cprogramming.com/begin.html}{\ul{http://www.cprogramming.com/begin.html}}
  \item
    \href{http://www.learn-c.org}{\ul{http://www.learn-c.org}}
  \item
    Accelerated C++: Practical Programming by Example
  \end{itemize}
\item
  Ghiduri de stil

  \begin{itemize}
  \item
    \href{http://users.ece.cmu.edu/~eno/coding/CCodingStandard.html}{\ul{http://users.ece.cmu.edu/\textasciitilde eno/coding/CCodingStandard.html}}
  \item
    \href{http://www.cs.swarthmore.edu/~newhall/unixhelp/c_codestyle.html}{\ul{http://www.cs.swarthmore.edu/\textasciitilde newhall/unixhelp/c\_codestyle.html}}
  \item
    \href{http://google-styleguide.googlecode.com/svn/trunk/cppguide.xml}{\ul{http://google-styleguide.googlecode.com/svn/trunk/cppguide.xml}}
  \end{itemize}
\end{itemize}

\subsection{\texorpdfstring{\textbf{Profiler}}{Profiler}}\label{profiler}
Biblioteca care se va utiliza pentru generarea graficelor, fiecare student va trebui sa parcurgă exemplul și tutorialul dat.
\\\\
Cea mai recentă versiune se găseste aici: 

https://github.com/cypryoprisa/utcn-fa-profiler
\\\\
Profiler Tutorial:

- \href{https://youtu.be/iE4FFwdncDk}{Part 1}

- \href{https://youtu.be/vMhwBXEelSA}{Part 2}

- \href{https://youtu.be/ZOztj7aPWs4}{Part 3}




\end{document}