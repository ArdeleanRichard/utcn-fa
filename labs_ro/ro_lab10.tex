\documentclass[../ro-fa-lab.tex]{subfiles}
\usepackage{hyperref}
\hypersetup{
    pdftitle={(RO) L10 - DFS},   % The title shown in the browser tab
    pdfauthor={},         % Your name or organization
    pdfsubject={},   % A brief description
    pdfkeywords={}
}

\begin{document}


\section{\texorpdfstring{\textbf{Tema Nr. 10: Căutare în adâncime (DFS)}}{Tema Nr. 10: Căutare în adâncime (DFS)}}\label{assign9}

\textbf{Tema Nr. 10: Căutare în adâncime (DFS)}

\textbf{Timp Alocat}: 2 ore

\subsection{Implementare}\label{implementare}

Se cere implementarea corectă și eficientă a algoritmului de căutare în
adâncime (Depth-First Search - DFS) (\emph{Capitolul
22.3\citep{cormen}}). Pentru reprezentarea grafurilor, va trebui să
folosești liste de adiacență. De asemenea, va trebui să:

\begin{itemize}
\item
  Implementarea algoritmului Tarjan pentru componente tare conexe
\item
  Implementezi sortarea topologică (\emph{Capitolul
  22.4}\emph{\citep{cormen}})
\end{itemize}

\subsection{Cerințe}\label{cerinux21be}

\subsubsection{DFS (5p)}\label{dfs-5p}

Demonstrați corectitudinea algoritmului pe un graf de dimensiune mică:

\begin{itemize}
\item
  afișați graful inițial (liste de adiacență)
\item
  afișați arborele rezultat în urma DFS
\end{itemize}

\subsubsection{Sortare topologică (1p)}\label{sortare-topologicux103-1p}

Demonstrați corectitudinea algoritmului pe un graf de dimensiune mică:

\begin{itemize}
\item
  afișați graful inițial (liste de adiacență)
\item
  afișați listă de noduri sortate topologic (dacă are / dacă nu are de
  ce nu are?)
\end{itemize}

\subsubsection{Tarjan (2p)}\label{tarjan-2p}

Demonstrați corectitudinea algoritmului pe un graf de dimensiune mică:

\begin{itemize}
\item
  afișați graful inițial (liste de adiacență)
\item
  afișați componentele puternic conexe ale grafului
\end{itemize}

\subsubsection{Analiza performanței pentru DFS
(2p)}\label{analiza-performanux21bei-pentru-dfs-2p}

\textbf{!} Înainte de a începe să lucrați la partea de evaluare,
asigurați-vă că aveți un algoritm corect!

Cum timpul de execuție al algoritmului DFS variază în funcție de numărul
de vârfuri (\textbar V\textbar) și de numărul de muchii
(\textbar E\textbar) aveți de făcut următoarele analize:

\begin{enumerate}
\def\labelenumi{\arabic{enumi}.}
\item
  Fixați \textbar V\textbar=100 și variați \textbar E\textbar{} între
  1000 și 4500 cu un pas de 100. Generați pentru fiecare caz un graf
  aleator și asigurați-vă că nu generați aceeași muchie de 2 ori.
  Execută DFS pentru fiecare graf generat și numără operațiile
  efectuate. Apoi construiește graficul cu variația numărului de
  operații în funcție de \textbar E\textbar;
\item
  Fixați \textbar E\textbar=4500 și variați \textbar V\textbar{} între
  100 și 200 cu un pas de 10. Repetă procedura de mai sus și
  construiește graficul cu variația numărului de operații în funcție de
  \textbar V\textbar.
\end{enumerate}

\end{document}