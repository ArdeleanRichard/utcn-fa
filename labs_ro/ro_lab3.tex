\documentclass[../ro-fa-lab.tex]{subfiles}
\usepackage{hyperref}
\hypersetup{
    pdftitle={(RO) L3 - Heapsort vs Quicksort},   % The title shown in the browser tab
    pdfauthor={},         % Your name or organization
    pdfsubject={},   % A brief description
    pdfkeywords={}
}

\begin{document}


\section{\texorpdfstring{\textbf{Tema Nr. 3: Analiza și compararea metodelor avansate de sortare
-- HeapSort și QuickSort / QuickSelect}}{Tema Nr. 3: Analiza și compararea metodelor avansate de sortare
-- HeapSort și QuickSort / QuickSelect}}\label{assign3}


\textbf{Timp alocat:} 2 ore

\subsection{Implementare}\label{implementare}

Se cere implementarea \textbf{corectă} și \textbf{eficientă} a Sortării
Rapide (\emph{Quicksort}), Sortării Rapide Hibridizate (\emph{Hybrid
Quicksort}) și \emph{Quick-Select} (\emph{Randomized-Select}). Se cere
și analizarea comparative a complexității Sortării folosind Heap-uri
\emph{(Heapsort,} implementat în Tema Nr. 2\emph{)} și Sortarea Rapidă
\emph{(Quicksort)}.

Informații utile și pseudo-cod găsiți în notițele de curs
sau în carte\citep{cormen}:

\begin{itemize}
\item
  \emph{Heapsort}: capitolul 6 (Heapsort)
\item
  \emph{Quicksort}: capitolul 7 (Quicksort)
\item
  \emph{Hibridizare quicksort utilizând insertion sort
  iterativ -} in quicksort, pentru dimensiuni de șir \textless{} prag,
  se utilizeaza insertion sort (folosiți implementarea insertion sort
  din Tema Nr. 1).
\item
  \emph{QuickSelect/Randomized Select:} capitolul 9
\end{itemize}

\subsection{Cerințe minimale pentru
notare}\label{cerinux21be-minimale-pentru-notare}

\begin{itemize}
\item
  Interpretați graficul și notați observațiile personale în antetul
  fișierului \emph{main.cpp}, într-un comentariu bloc informativ.
\end{itemize}

\begin{itemize}
\item
  Pregătiți un exemplu pentru exemplificarea corectitudinii fiecărui
  algoritm implementat.
\item
  Nu preluăm teme care nu sunt indentate și care nu sunt organizate în
  funcții (de exemplu nu preluam teme unde tot codul este pus in main)
\end{itemize}

\begin{itemize}
\item
  \emph{\textbf{Punctajele din barem se dau pentru rezolvarea corectă și
  completă a cerinței, calitatea interpretărilor din comentariul bloc și
  răspunsul corect dat de voi la întrebările puse de către profesor.}}
\end{itemize}

\subsection{Cerințe}\label{cerinux21be}

\subsubsection{QuickSort: implementare
(2p)}\label{quicksort-implementare-2p}

Corectitudinea algoritmului va trebui exemplificată pe date de intrare
de dimensiuni mici.

\subsubsection{QuickSort: evaluare în caz mediu statistic, favorabil si
defavorabil
(3p)}\label{quicksort-evaluare-uxeen-caz-mediu-statistic-favorabil-si-defavorabil-3p}

\textbf{!} Înainte de a începe să lucrați pe partea de evaluare,
asigurați-vă că aveți un \textbf{algoritm corect}!

Pașii de analiză:

- variați dimensiunea șirului de intrare (\emph{n}) între
{[}100\ldots10000{]}, cu un increment de maxim 500 (sugerăm 100).

- pentru fiecare dimensiune (\emph{n}), generați date de intrare
adecvate metodei de construcție; rulați metoda numărând operațiile
elementare (atribuiri și comparații - pot fi numărate împreună pentru
această temă).

\textbf{!} Doar atribuirile și comparație care se fac pe datele de
intrare și pe datele auxiliare corespunzătoare se iau în considerare.

\subsubsection{QuickSort și HeapSort: analiză comparativă a cazului mediu
statistic
(2p)}\label{quicksort-ux219i-heapsort-analizux103-comparativux103-a-cazului-mediu-statistic-2p}

Se cere compararea celor două metode de sortare în cazul \textbf{mediu
statistic}. Pentru cazul \textbf{mediu statistic} va trebui să repetați
măsurătorile de \emph{m} ori (m=5) și să raportați valoarea lor medie;
de asemenea, pentru cazul \textbf{mediu statistic}, asigurați-vă că
folosiți \textbf{aceleași} date de intrare pentru cele două metode.

Generați un grafic ce compară cele două metode de construcție în cazul
\textbf{mediu statistic} pe baza numărului de operații obținut la pasul
anterior.

Dacă o curba nu poate fi vizualizată corect din cauza că celelalte curbe
au o rată mai mare de creștere, atunci plasați noua curbă pe un alt
grafic. Denumiți adecvat graficele și curbele.

\subsubsection{Implementarea hibridizării quicksort-ului
(1p)}\label{implementarea-hibridizux103rii-quicksort-ului-1p}

Corectitudinea algoritmilor va trebui exemplificată pe date de intrare
de dimensiuni mici.

\subsubsection{Determinare a unui prag optim în hibridizare + motivație
(grafice/măsuratori)
(1p)}\label{determinare-a-unui-prag-optim-uxeen-hibridizare-motivaux21bie-graficemux103suratori-1p}

Determinarea optimului din perspectiva pragului utilizat se realizează
prin varierea valorii de prag pentru care se aplică insertion sort.

Comparați rezultatele obținute din perspectiva performanței (timpului de
execuție și a numărului de operații) pentru a determina o valoare optima
a pragului. Puteți folosi 10,000 ca dimensiune fixă a vectorului ce
urmează sa fie sortat și variați pragul între {[}5,50{]} cu un increment
de 1 până la 5.

Numărul de teste care trebuie sa fie repetate (nr\_tests din exemplul de
mai sus) trebuie ales în funcție de procesor și modul de compilare.
Sugerăm valori mai mari, precum 100 sau 1000.

\subsubsection{\texorpdfstring{Analiză comparativă (între \emph{quicksort
și quicksort hibridizat}) din punct de vedere a numărului de operații și
a timpului efectiv de execuție
(1p)}{Analiză comparativă (între quicksort și quicksort hibridizat) din punct de vedere a numărului de operații și a timpului efectiv de execuție (1p)}}\label{analizux103-comparativux103-uxeentre-quicksort-ux219i-quicksort-hibridizat-din-punct-de-vedere-a-numux103rului-de-operaux21bii-ux219i-a-timpului-efectiv-de-execuux21bie-1p}

\textbf{!} Înainte de a începe să lucrați pe partea de evaluare,
asigurați-vă că aveți un \textbf{algoritm corect}!

Corectitudinea algoritmilor va trebui exemplificată pe date de intrare
de dimensiuni mici.

Pentru hibridizare quicksort, trebuie să utilizați versiunea iterativă
de insertion sort din prima temă în cazul în care dimensiunea vectorului
este mică (sugerăm utilizarea insertion sort dacă vectorul are sub 30 de
elemente). Comparați \emph{timpul de rulare și numărul de operații}
(asignări + comparații) pentru quicksort implementat în tema 3 cu cel
hibridizat.

Pentru a măsura timpul de execuție puteți folosi Profiler similar cu
exemplul de mai jos.
\\

\emph{profiler.startTimer("your\_function", current\_size);}

\emph{for(int test=0; test\textless nr\_tests; ++test) \{}

\quad\emph{your\_function(array, current\_size);}

\emph{\}}

\emph{profiler.stopTimer("your\_function", current\_size);}
\\

În momentul în care măsurați timpul de execuție, asigurați-vă că opriți
orice alte procese care nu sunt necesare.

\subsubsection{Bonus: QuickSelect - Randomized-Select
(0.5p)}\label{bonus-quickselect---randomized-select-0.5p}

Corectitudinea algoritmului va trebui exemplificată pe date de intrare
de dimensiuni mici.

Pentru QuickSelect (Randomized-Select) nu trebuie facuta analiza
complexității, doar corectitudinea trebuie exemplificată.

\end{document}