\documentclass[../ro-fa-lab.tex]{subfiles}

\usepackage{hyperref}
\hypersetup{
    pdftitle={(RO) L7 - Statistici dinamice de ordine},   % The title shown in the browser tab
    pdfauthor={},         % Your name or organization
    pdfsubject={},   % A brief description
    pdfkeywords={}
}

\begin{document}


\section{\texorpdfstring{\textbf{Tema Nr. 7: Statistici dinamice de ordine}}{Tema Nr. 7: Statistici dinamice de ordine}}\label{assign7}


\textbf{Timp alocat:} 2 ore

\subsection{Implementare}\label{implementare}

Se cere implementarea \textbf{corectă} și \textbf{eficientă} a
operațiilor de management ale unui \textbf{arbore de statistică de
ordine} (\emph{capitolul 14 din carte}\footnote{}).

Se cere să folosiți un \emph{arbore binar} de \emph{căutare}
\textbf{perfect} \textbf{echilibrat}. Fiecare nod din arbore trebuie
extins cu un câmp \emph{size} (dimensiunea sub-arborelui ce are nodul ca
rădăcină).

Operațiile de management ale unui \textbf{arbore de statistică de
ordine}:

\begin{itemize}
\item
  BUILD\_TREE(n)

  \begin{itemize}
  \item
    \emph{construiește} un arbore binar de căutare \textbf{echilibrat}
    cu cheile 1,2,...,n (\emph{hint}: \emph{divide et impera})
  \item
    nu uitați să inițializați câmpul \emph{size}
  \end{itemize}
\item
  OS-SELECT(tree, i)

  \begin{itemize}
  \item
    selectează elementul cu a \emph{i}-a cea mai mică cheie
  \item
    pseudocodul poate fi găsit la \emph{Capitolul 14.1 din
    carte\citep{cormen}}
  \end{itemize}
\item
  OS-DELETE(tree, i)

  \begin{itemize}
  \item
    puteți folosi ștergerea dintr-un arbore binar de căutare, fără a
    crește înălțimea arborelui (De ce nu trebuie să re-balansați
    arborele?)
  \item
    nu uitați să păstrați câmpul \emph{size} consistent o dată cu
    ștergerile din arbore
  \item
    există mai multe abordări prin care puteți modifica câmpul
    \emph{size} fără a crește complexitatea algoritmului (găsiți cea mai
    bună soluție)
  \end{itemize}
\end{itemize}

Seamănă OS-SELECT cu ceva ce ați studiat în acest semestru?

\subsection{Cerințe}\label{cerinux21be}

\subsubsection{BUILD\_TREE: implementare corectă și eficientă
(5p)}\label{build_tree-implementare-corectux103-ux219i-eficientux103-5p}

Corectitudinea algoritmilor va trebui demonstrată pe date de intrare de
dimensiuni mici (11)

\begin{itemize}
\item
  afișați (cu pretty print) arborele construit inițial
\end{itemize}

\subsubsection{OS\_SELECT: implementare corectă și eficientă
(1p)}\label{os_select-implementare-corectux103-ux219i-eficientux103-1p}

Corectitudinea algoritmilor va trebui demonstrată pe date de intrare de
dimensiuni mici (11)

\begin{itemize}
\item
  executați OS-SELECT pentru câțiva (cel puțin 3) indecși selectați
  aleator.
\end{itemize}

\subsubsection{OS\_DELETE: implementare corectă și eficientă
(2p)}\label{os_delete-implementare-corectux103-ux219i-eficientux103-2p}

Corectitudinea algoritmilor va trebui demonstrată pe date de intrare de
dimensiuni mici (11)

\begin{itemize}
\item
  executați secvența OS-SELECT urmat de OS-DELETE pentru câțiva (cel
  puțin 3) indecși selectați aleator (3) și \emph{afișati arborele dupa
  fiecare execuție}.
\end{itemize}

\subsubsection{Evaluarea operațiilor de management - BUILD, SELECT, DELETE
(2p)}\label{evaluarea-operaux21biilor-de-management---build-select-delete-2p}

\textbf{!} Înainte de a începe să lucrați pe partea de evaluare,
asigurați-vă că aveți un \textbf{algoritm corect}!

După ce sunteți siguri că algoritmul funcționează corect:

\begin{itemize}
\item
  variați \emph{n} de la 100 la 10000 cu un pas de 100;
\item
  pentru fiecare \emph{n} (nu uitați să repetați de 5 ori)

  \begin{itemize}
  \item
    construiți (BUILD) arborele cu elemente de la \emph{1 la n}
  \item
    repetați de \emph{n} ori secvența OS-SELECT urmat OS-DELETE folosind
    un index selectat aleator dintre elementele rămase în arbore
  \item
    Evaluați numărul de operații necesare pentru fiecare operație de
    management (BUILD, SELECT, DELETE \emph{-- reprezentați rezultatele
    sub forma unui grafic cu trei serii}). Evaluați complexitatea
    operațiilor de management ca și suma atribuirilor și a comparațiilor
    pentru fiecare valoare a lui \emph{n}.
  \end{itemize}
\end{itemize}

\subsubsection{Bonus: Implementarea utilizând AVL / arbori roșu și negru
(1p)}\label{bonus-implementarea-utilizuxe2nd-avl-arbori-roux219u-ux219i-negru-1p}

\end{document}