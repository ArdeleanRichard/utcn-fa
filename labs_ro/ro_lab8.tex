\documentclass[../ro-fa-lab.tex]{subfiles}

\usepackage{hyperref}
\hypersetup{
    pdftitle={(RO) L8 - Multimi disjuncte},   % The title shown in the browser tab
    pdfauthor={},         % Your name or organization
    pdfsubject={},   % A brief description
    pdfkeywords={}
}

\begin{document}


\section{\texorpdfstring{\textbf{Tema Nr. 8: Mulțimi disjuncte}}{Tema Nr. 8: Mulțimi disjuncte}}\label{assign8}

\textbf{Timp alocat:} 2 ore

\subsection{Implementare}\label{implementare}

Se cere implementarea \textbf{corectă} și \textbf{eficientă} a
operațiilor de bază pe \textbf{mulțimi disjuncte} (\emph{capitolul
21.1}\footnote{Thomas H. Cormen, Charles E. Leiserson, Ronald L. Rivest
  and Clifford Stein. \emph{Introduction to Algorithms}}) și a
algoritmului lui \textbf{Kruskal} (găsirea arborelui de acoperire
minimă, \emph{capitolul 23.2}\citep{cormen}) folosind mulțimi
disjuncte.

Se cere să folosiți o pădure de arbori pentru reprezentarea mulțimilor
disjuncte. Fiecare arbore trebuie extins cu un câmp \emph{rank}
(înălțimea arborelui).

Operațiile de bază pe \textbf{mulțimi disjuncte} sunt\textbf{:}

\begin{itemize}
\item
  MAKE\_SET (x)

  \begin{itemize}
  \item
    creează o mulțime nouă ce conține elementul \emph{x}
  \end{itemize}
\item
  UNION (x, y)

  \begin{itemize}
  \item
    realizează reuniunea dintre mulțimea care îl conține pe \emph{x} și
    mulțimea care îl conține pe y
  \item
    euristica \emph{union by rank} ține cont de înălțime celor doi
    arbori pentru a realiza reuniunea dintre mulțimi
  \item
    pseudocodul poate fi găsit la \emph{capitolul
    21.3}\citep{cormen}
  \end{itemize}
\item
  FIND\_SET (x)

  \begin{itemize}
  \item
    caută mulțime în care se află \emph{x}
  \item
    euristica \emph{path compression} leagă toate elementele de pe
    ramura cu \emph{x} la rădăcina arborelui
  \end{itemize}
\end{itemize}

\subsection{Cerințe}\label{cerinux21be}

\subsubsection{Implementare corectă a MAKE\_SET, UNION și FIND\_SET
(5p)}\label{implementare-corectux103-a-make_set-union-ux219i-find_set-5p}

Corectitudinea algoritmilor va trebui demonstrată pe date de intrare de
dimensiuni mici

\begin{itemize}
\item
  creați (MAKE) 10 mulțimi + afișare conținuturilor seturilor
\item
  executați secvența UNION și FIND\_SET pentru 5 elemente + afișare
  conținuturilor seturilor
\end{itemize}

\subsubsection{Implementarea corectă și eficientă a algoritmului lui
Kruskal
(2p)}\label{implementarea-corectux103-ux219i-eficientux103-a-algoritmului-lui-kruskal-2p}

Corectitudinea algoritmului va trebui demonstrată pe date de intrare de
dimensiuni mici

\begin{itemize}
\item
  creați un graf cu 5 noduri și 9 muchii + \textbf{afișare muchii}
\item
  aplicarea algoritmului lui Kruskal + \textbf{afișarea muchiilor alese}
\end{itemize}

\subsubsection{Evaluarea operațiilor pe mulțimi disjuncte (MAKE, UNION,
FIND) folosind algoritmului lui Kruskal
(3p)}\label{evaluarea-operaux21biilor-pe-mulux21bimi-disjuncte-make-union-find-folosind-algoritmului-lui-kruskal-3p}

\textbf{!} Înainte de a începe să lucrați pe partea de evaluare,
asigurați-vă că aveți un \textbf{algoritm corect}!

O dată ce sunteți siguri că algoritmul funcționează corect:

\begin{itemize}
\item
  variați \emph{n} de la 100 la 10000 cu un pas de 100
\item
  pentru fiecare \emph{n}

  \begin{itemize}
  \item
    construiți un graf \textbf{conex}, \textbf{neorientat} și
    \textbf{aleatoriu} cu ponderi pe muchii (\textbf{n} noduri,
    \textbf{n*4} muchii)
  \item
    determinați arborele de acoperire minima folosind algoritmul lui
    Kruskal

    \begin{itemize}
    \item
      evaluați efortul computațional \ul{al fiecărei operații de bază}
      (MAKE, UNION, FIND \emph{-- reprezentați rezultatele sub forma
      unui grafic cu trei serii}) pe mulțimi disjuncte ca suma
      comparațiilor și atribuțiilor efectuate; astfel, ar trebui să
      existe \textbf{3 serii în grafic}, câte una pentru fiecare
      operație.
    \end{itemize}
  \end{itemize}
\end{itemize}

\end{document}