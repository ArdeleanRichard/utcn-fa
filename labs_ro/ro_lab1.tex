\documentclass[../ro-fa-lab.tex]{subfiles}

\usepackage{hyperref}
\hypersetup{
    pdftitle={(RO) L1 - Metode directe de sortare},   % The title shown in the browser tab
    pdfauthor={},         % Your name or organization
    pdfsubject={},   % A brief description
    pdfkeywords={}
}

\begin{document}


\section{\texorpdfstring{\textbf{Tema nr. 1: Analiza și Compararea Metodelor Directe de Sortare}}{Tema nr. 1: Analiza și Compararea Metodelor Directe de Sortare}}\label{assign1}

\textbf{Timp alocat:} 2 ore

\subsection{Implementare}\label{implementare}

Se cere implementarea \textbf{corectă} și \textbf{eficientă} a 3 metode
directe de sortare (\emph{Sortarea Bulelor}, \emph{Sortarea prin
Inserție} -- folosind inserție liniară sau binară, și \emph{Sortarea
prin Selecție})

\begin{itemize}
\item
  Intrare: un șir de numere \(< a_{1},a_{2},\ldots,a_{n} >\)
\item
  Ieșire: o permutare ordonată a șirului de la intrare
  \({< a}_{1}' \leq a_{2}' \leq \ldots \leq a_{n}' >\)
\end{itemize}

Toate informațiile necesare și pseudo-codul se găsesc în notițele de la
\textbf{Seminarul nr. 1} (Sortarea prin Inserție este prezentată și în
\textbf{carte\citep{cormen} -- secțiunea 2.1}). Să verificați că
ați implementat varianta eficientă pentru fiecare din algoritmii de
sortare (dacă mai multe versiuni au fost prezentate)

\subsection{Cerințe minimale pentru
notare}\label{cerinux21be-minimale-pentru-notare}

\begin{itemize}
\item
  Interpretați graficul și notați observațiile personale în antetul
  fișierului \emph{main.cpp}, într-un comentariu bloc informativ.
\end{itemize}

\begin{itemize}
\item
  Pregatiti un exemplu pentru exemplificarea corectitudinii fiecărui
  algoritm implementat
\item
  Nu preluăm teme care nu sunt indentate și care nu sunt organizate în
  funcții (de exemplu nu preluam teme unde tot codul este pus in main)
\item
  \emph{\textbf{Punctajele din barem se dau pentru rezolvarea corectă și
  completă a cerinței, calitatea interpretărilor din comentariul bloc și
  răspunsul corect dat de voi la întrebările puse de către profesor.}}
\end{itemize}

\subsection{Cerințe}\label{cerinux21be}

\subsubsection{Implementarea metodelor de sortare
(5.5p)}\label{implementarea-metodelor-de-sortare-5.5p}

\begin{itemize}
\item
  Bubble sort (1.5p)
\item
  Insertion sort (2p)
\item
  Selection sort (2p)
\end{itemize}

Corectitudinea algoritmilor va trebui demonstrată pe un vector de
dimensiuni mici (care poate să fie codat în funcția \emph{main}).

\subsubsection{Analiză algoritmilor pentru caz mediu statistic (1.5p -- 0.5
per
algoritm)}\label{analizux103-algoritmilor-pentru-caz-mediu-statistic-1.5p-0.5-per-algoritm}

\textbf{!} Înainte de a începe să lucrați pe partea de evaluare a
complexitatii algoritmilor, asigurati-va că aveti un algoritm corect!

Se cere compararea celor 3 algoritmi în cazurile: \textbf{favorabil
(best)}, \textbf{mediu statistic (average)} și \textbf{defavorabil
(worst)}. Pentru cazul \textbf{mediu} va trebui să repetați măsurătorile
de \textbf{m} ori (m=5 este suficient) și să raportați media
rezultatelor; de asemenea, pentru cazul \textbf{mediu,} asigurați-vă că
folosiți \textbf{aceleași} date de intrare pentru cele 3 metode de
sortare (astfel încât compararea lor să fie corectă); identificați și
generați date de intrare pentru cazurile: \textbf{favorabil} și
\textbf{defavorabil}, pentru toate cele 3 metode de sortare.

Pașii de analiză ai metodelor de sortare pentru fiecare din cele 3
cazuri (\textbf{favorabil, defavorabil, mediu)}:

- variați dimensiunea șirului de la intrare (\emph{n}) între
{[}100\ldots10.000{]}, cu un increment de maxim 500 (sugerăm 100).

- pentru fiecare dimensiune, generati datele de intrare adecvate pentru
metoda de sortare; rulați metoda de sortare numărând operațiile (numărul
de atribuiri, numărul de comparații și suma lor).

\begin{quote}
\textbf{!} Doar atribuirile (=) și comparațiile (\textless, ==,
\textgreater, !=) care se fac pe datele de intrare și pe datele
auxiliare corespunzătoare se iau în considerare.
\end{quote}

\subsubsection{Analiză in caz favorabil și defavorabil (3p - câte 0.5p
pentru fiecare caz a fiecărui
algoritm)}\label{analizux103-in-caz-favorabil-ux219i-defavorabil-3p---cuxe2te-0.5p-pentru-fiecare-caz-a-fiecux103rui-algoritm}

Pentru fiecare caz de analiză (\textbf{favorabil, defavorabil} si
\textbf{mediu),} generati grafice care compara cele 3 metode de sortare;
folosiți grafice diferite pentru numărul de atribuiri, comparații și
suma lor. Dacă o curbă nu poate fi vizualizată corect din cauza că
celelalte curbe au o rată mai mare de creștere (ex: o funcție liniară
pare constantă atunci când este plasată în același grafic cu o funcție
pătratică), atunci plasați noua curbă și pe un alt grafic. Denumiti
adecvat graficele și curbele.

Corectitudinea algoritmilor va trebui demonstrată pe un vector de
dimensiuni mici (care poate să fie codat în funcția ''main'').

\subsubsection{Bonus: Insertion sort prin inserție binară
(0.5p)}\label{bonus-insertion-sort-prin-inserux21bie-binarux103-0.5p}

Corectitudinea algoritmilor va trebui demonstrată pe un vector de
dimensiuni mici (care poate să fie codat în funcția ''main'').

Va trebui să comparați insertion sort prin inserție binară cu toți
ceilalți algoritmi (bubble, insertion, selection) pentru toate cazurile
(favorabil, mediu statistic, defavorabil).

\end{document}