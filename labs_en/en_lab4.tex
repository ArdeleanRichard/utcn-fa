\documentclass[../en-fa-lab.tex]{subfiles}


\usepackage{hyperref}
\hypersetup{
    pdftitle={(EN) L4 - Merge k lists},   % The title shown in the browser tab
    pdfauthor={},         % Your name or organization
    pdfsubject={},   % A brief description
    pdfkeywords={}
}

\begin{document}

\section{\texorpdfstring{\textbf{Assignment No. 4: Merge k Ordered Lists Efficiently}}{Assignment No. 4: Merge k Ordered Lists Efficiently}}\label{assign4}

\textbf{Allocated time:} 2 hours

\subsection{Implementation}\label{implementation}

You are required to implement \textbf{correctly} and
\textbf{efficiently} an O(nlogk) method for \textbf{merging k sorted
sequences}, where n is the total number of elements. (Hint: use a heap,
see \emph{Seminar no. 2} notes).

Implementation requirements:

\begin{itemize}
\item
  Use linked lists to represent the \emph{k} sorted sequences and the
  output sequence
\end{itemize}

Input: \emph{k} lists of numbers
\(< a_{1}^{i},a_{2}^{i},\ldots,a_{m_{i}}^{i} >\),
\(\sum_{i = 1}^{k}m_{i} = n\)

Output: a permutation of the union of the input sequences:
\(a_{1}' \leq a_{2}' \leq \ldots \leq a_{n}'\)

\subsection{Minimal requirements for
grading}\label{minimal-requirements-for-grading}

\begin{itemize}
\item
  Interpret the chart and write your observations in the header (block
  comments) section at the beginning of your \emph{main.cpp} file.
\end{itemize}

\begin{itemize}
\item
  Prepare a demo for each algorithm implemented.
\item
  We do not accept assignments without code indentation and with code
  not organized in functions (for example where the entire code is in
  the main function).
\item
  \emph{\textbf{The points from the requirements correspond to a correct
  and complete solution, quality of interpretation from the block
  comment and the correct answer to the questions from the teacher.}}
\end{itemize}

\subsection{Requirements}\label{requirements}

\subsubsection{\texorpdfstring{Generate \emph{k} randomly sized sorted
lists (having \emph{n} elements in total, \emph{n} and \emph{k} given as
parameters) and the merging of 2 lists
(5p)}{Generate k randomly sized sorted lists (having n elements in total, n and k given as parameters) and the merging of 2 lists (5p)}}\label{generate-k-randomly-sized-sorted-lists-having-n-elements-in-total-n-and-k-given-as-parameters-and-the-merging-of-2-lists-5p}

You will have to show your algorithm (\emph{generation and merging})
works on a small-sized input (e.g. k=4, n=20).

\subsubsection{\texorpdfstring{Adapt \emph{min-heap} operations to work on
the new structure and the merging of k lists
(3p)}{Adapt min-heap operations to work on the new structure and the merging of k lists (3p)}}\label{adapt-min-heap-operations-to-work-on-the-new-structure-and-the-merging-of-k-lists-3p}

You will have to show your algorithm (\emph{merging}) works on a
small-sized input (e.g. k=4, n=20).

\subsubsection{Evaluation of the algorithm in average case
(2p)}\label{evaluation-of-the-algorithm-in-average-case-2p}

\textbf{!} Before you start to work on the algorithm\textquotesingle s
evaluation code, make sure you have a correct algorithm!

We will make the \textbf{average case} analysis of the algorithm.
Remember that, in the average case, you have to repeat the measurements
several times. Since both \textbf{k} and \textbf{n} may vary, we will
make each analysis in turn:

\begin{itemize}
\item
  Choose, in turn, 3 constant values for \emph{k} (\textbf{k1=5, k2=10,
  k3=100}); generate \emph{k} \textbf{random} sorted lists for each
  value of \emph{k} so that the number of elements in all the lists
  \emph{n} varies between \textbf{100 and 10000}, with a maximum
  increment of 400 (we suggest 100); run the algorithm for all values of
  \emph{n} (for each value of \emph{k}); generate a chart that
  represents the \textbf{sum of assignments and comparisons} done by the
  merging algorithm for each value of \emph{k} as a curve (total 3
  curves).
\item
  Set \emph{\textbf{n = 10.000}}; the value of \emph{k} must vary
  between 10 and 500 with an increment of 10; generate \emph{k}
  \textbf{random} sorted lists for each value of \emph{k} so that the
  number of elements in all the lists is 10000; test the merging
  algorithm for each value of \emph{k} and generate a chart that
  represents the \textbf{sum of assignments and comparisons} as a curve.
\end{itemize}

\end{document}