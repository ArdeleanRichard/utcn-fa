\documentclass[../en-fa-lab.tex]{subfiles}

\usepackage{hyperref}

\hypersetup{
    pdftitle={(EN) L7 - Dynamic order statistics},   % The title shown in the browser tab
    pdfauthor={},         % Your name or organization
    pdfsubject={},   % A brief description
    pdfkeywords={}
}

\begin{document}

\section{\texorpdfstring{\textbf{Assignment No. 7: Dynamic Order Statistics}}{Assignment No. 7: Dynamic Order Statistics}}\label{assign7}

\textbf{Allocated time:} 2 hours

\subsection{Implementation}\label{implementation}

You are required to implement \textbf{correctly} and
\textbf{efficiently} the management operations of an \textbf{order
statistics tree} (\emph{chapter 14.1 from the book\citep{cormen}}).

You have to use a balanced, augmented Binary Search Tree. Each node in
the tree holds, besides the necessary information, also the \emph{size}
field (i.e. the size of the sub-tree rooted at the node).

The management operations of an \textbf{order statistics tree} are:

\begin{itemize}
\item
  BUILD\_TREE(n)

  \begin{itemize}
  \item
    \emph{builds} a \textbf{balanced} BST containing the keys 1,2,...n
    (\emph{hint:} use a divide and conquer approach)
  \item
    make sure you initialize the size field in each tree node
  \end{itemize}
\item
  OS-SELECT(tree, i)

  \begin{itemize}
  \item
    selects the element with the \emph{i}-th smallest key
  \item
    the pseudo-code is available in \emph{chapter 14.1 from the
    book\citep{cormen}}
  \end{itemize}
\item
  OS-DELETE(tree, i)

  \begin{itemize}
  \item
    you may use the deletion from a BST, without increasing the height
    of the tree (why don't you need to rebalance the tree?)
  \item
    keep the size information consistent after subsequent deletes
  \item
    there are several alternatives to update the size field without
    increasing the complexity of the algorithm (it is up to you to
    figure this out).
  \end{itemize}
\end{itemize}


Does OS-SELECT resemble anything you studied this semester?

\begin{center}\rule{0.5\linewidth}{0.5pt}\end{center}

\subsection{Requirements}\label{requirements}

\subsubsection{BUILD\_TREE: correct and efficient implementation
(5p)}\label{build_tree-correct-and-efficient-implementation-5p}

You will have to prove your algorithm(s) work on a small-sized input
(11)

\begin{itemize}
\item
  pretty-print the initially built tree
\end{itemize}

\subsubsection{OS\_SELECT: correct and efficient implementation
(1p)}\label{os_select-correct-and-efficient-implementation-1p}

You will have to prove your algorithm(s) work on a small-sized input
(11)

\begin{itemize}
\item
  execute OS-SELECT for a few elements (at least 3) by a randomly
  selected index
\end{itemize}

\subsubsection{OS\_DELETE: correct and efficient implementation
(2p)}\label{os_delete-correct-and-efficient-implementation-2p}

You will have to prove your algorithm(s) work on a small-sized input
(11)

\begin{itemize}
\item
  execute OS-SELECT followed by OS-DELETE for a few elements (at least
  3) by a randomly selected index \emph{and pretty-print the tree after
  each execution}.
\end{itemize}

\subsubsection{Management operations evaluation - BUILD, SELECT, DELETE
(2p)}\label{management-operations-evaluation---build-select-delete-2p}

\textbf{!} Before you start to work on the algorithms evaluation code,
make sure you have a \textbf{correct algorithm}!

Once you are sure your program works correctly:

\begin{itemize}
\item
  vary \emph{n} from 100 to 10000 with a step of 100;
\item
  for each n (don't forget to repeat 5 times),

  \begin{itemize}
  \item
    BUILD a tree with elements from \emph{1 to n}
  \item
    perform \emph{n} sequences of OS-SELECT and OS-DELETE operations
    using a randomly selected index based on the remaining number of
    elements in the BST,
  \item
    Evaluate the number of operations needed for each management
    operation (BUILD, SELECT, DELETE \emph{-- resulting in a plot with 3
    series}). Evaluate the computational effort as the sum of the
    comparisons and assignments performed by each individual management
    operation ofr each value of \emph{n}.
  \end{itemize}
\end{itemize}

\subsubsection{Bonus: Implementation using AVL / Red black tree
(1p)}\label{bonus-implementation-using-avl-red-black-tree-1p}

\end{document}