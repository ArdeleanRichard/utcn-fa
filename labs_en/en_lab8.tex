\documentclass[../en-fa-lab.tex]{subfiles}

\usepackage{hyperref}

\hypersetup{
    pdftitle={(EN) L8 - Disjoint sets},   % The title shown in the browser tab
    pdfauthor={},         % Your name or organization
    pdfsubject={},   % A brief description
    pdfkeywords={}
}

\begin{document}

\section{\texorpdfstring{\textbf{Assignment No. 8: Disjoint Sets}}{Assignment No. 8: Disjoint Sets}}\label{assign8}
\textbf{}

\textbf{Allocated time:} 2 hours

\subsection{Implementation}\label{implementation}

You are required to implement \textbf{correctly} and
\textbf{efficiently} the base operations for \textbf{disjoint set}
(\emph{chapter 21.1}\footnote{Thomas H. Cormen, Charles E. Leiserson,
  Ronald L. Rivest and Clifford Stein. \emph{Introduction to Algorithms}})
and the \textbf{Kruskal's algorithm} (searching for the minimum spanning
tree, \emph{chapter 23.2}\citep{cormen}) using disjoint sets.

You have to use a tree as the representation of a disjoint set. Each
tree holds, besides the necessary information, also the \emph{rank}
field (i.e. the height of the tree).

The base operations on \textbf{disjoints sets} are:

\begin{itemize}
\item
  MAKE\_SET (x)

  \begin{itemize}
  \item
    creates a set with the element \emph{x}
  \end{itemize}
\item
  UNION (x, y)

  \begin{itemize}
  \item
    makes the union between the set that contains the element \emph{x}
    and the set that contains the element \emph{y}
  \item
    the heuristic \emph{union by rank} takes into account the height of
    the two trees so as to make the union
  \item
    the pseudo-code can be found in \emph{chapter
    21.3}\citep{cormen}
  \end{itemize}
\item
  FIND\_SET (x)

  \begin{itemize}
  \item
    searches for the set that contains the element \emph{x}
  \item
    the heuristic \emph{path compression} links all nodes that were
    found on the path to \emph{x} to the root node
  \end{itemize}
\end{itemize}

\subsection{Requirements}\label{requirements}

\subsubsection{Correct implementation of MAKE\_SET, UNION and FIND\_SET
(5p)}\label{correct-implementation-of-make_set-union-and-find_set-5p}

The correctness of the algorithm must be proved on a small-sized input

\begin{itemize}
\item
  create (MAKE) 10 sets + show the contents of the sets
\item
  execute the sequence UNION and FIND\_SET for 5 elements + show the
  contents of the sets
\end{itemize}

\subsubsection{Correct and efficient implementation for Kruskal's algorithm
(2p)}\label{correct-and-efficient-implementation-for-kruskals-algorithm-2p}

The correctness of the algorithm must be proved on a small-sized input

\begin{itemize}
\item
  create a graph of 5 nodes and 9 edges + \textbf{print the edges}
\item
  apply Kruskal's algorithm + \textbf{print the chosen edges}
\end{itemize}

\subsubsection{Evaluate the disjoint sets operations (MAKE, UNION, FIND)
using Kruskal's algorithm
(3p)}\label{evaluate-the-disjoint-sets-operations-make-union-find-using-kruskals-algorithm-3p}

\textbf{!} Before you start to work on the algorithms evaluation code,
make sure you have a correct algorithm!

Once you are sure your program works correctly:

\begin{itemize}
\item
  vary \emph{n} from 100 to 10000 with a step of 100
\item
  for each n

  \begin{itemize}
  \item
    build an \textbf{undirected}, \textbf{connected}, and
    \textbf{random} graph with random weights on edges (\textbf{n}
    nodes, \textbf{n*4} edges)
  \item
    find the minimum spanning tree using Kruskal's algorithm

    \begin{itemize}
    \item
      evaluate the computational effort of \ul{each individual base
      operation} (MAKE, UNION, FIND\emph{-- resulting in a plot with 3
      series}) on disjoint sets as the sum of the comparisons and
      assignments performed; thus, there should be \textbf{3 series in
      the plot}, one for each operation.
    \end{itemize}
  \end{itemize}
\end{itemize}

\end{document}