\documentclass[../ro-fa-lab.tex]{subfiles}

\usepackage{hyperref}
\hypersetup{
    pdftitle={(RO) L4 - Interclasare k liste},   % The title shown in the browser tab
    pdfauthor={},         % Your name or organization
    pdfsubject={},   % A brief description
    pdfkeywords={}
}

\begin{document}


\section{\texorpdfstring{\textbf{Tema Nr. 4: Interclasarea eficientă a \emph{k} liste ordonate}}{Tema Nr. 4: Interclasarea eficientă a k liste ordonate}}\label{assign4}

\textbf{Timp alocat:} 2 ore

\subsection{Implementare}\label{implementare}

Se cere implementarea \textbf{corectă} și \textbf{eficientă} a unei
metode de complexitate $O(n logk)$ pentru \textbf{interclasarea a \emph{k} liste
sortate}. Unde n este numărul total de elemente (Sugestie: folosiți un
heap, vezi notițele de la \emph{Seminarul al 2-lea}).

Cerințe de implementare:

\begin{itemize}
\item
  Folosiți liste înlănțuite pentru a reprezenta cele \emph{k} secvențe
  sortate și secvența de ieșire
\end{itemize}

Intrare: \emph{k} șiruri de numere
\(< a_{1}^{i},a_{2}^{i},\ldots,a_{m_{i}}^{i} >\),
\(\sum_{i = 1}^{k}m_{i} = n\)

Ieșire: o permutare a reuniunii șirurilor de la intrare
\(a_{1}' \leq a_{2}' \leq \ldots \leq a_{n}'\)





\subsection{Cerințe minimale pentru notare}\label{cerinux21be-minimale-pentru-notare}

Lipsa oricărei cerințe minimale (chiar și parțială) poate rezulta într-o notă mai mică prin penalizări sau refuzul de a prelua tema, rezultând în nota 0.

\begin{itemize}
\item
  \emph{Demo}: Pregătiți un exemplu pentru exemplificarea corectitudinii fiecărui
  algoritm implementat. Corectitudinea fiecărui algoritm se demonstrează printr-un exemplu simplu (maxim 10 valori).
\item
    Graficele create trebuie să fie ușor de evaluat, adică grupate și adunate prin funcțiile Profiler după cerințele temei. Tema nu va fi evaluată dacă conține o multitudine de grafice negrupate. De exemplu, analiza comparativă implică gruparea într-un singur grafic a algoritmilor comparați.
\item
  Interpretați graficul/graficele și notați observațiile personale în antetul
  fișierului \emph{main.cpp}, într-un comentariu bloc informativ.
\item
  Nu preluăm teme care nu sunt indentate și care nu sunt organizate în
  funcții (de exemplu, nu prelăum teme unde tot codul este pus în main).
\item
  \emph{Punctajele din barem sunt corespondente unei rezolvări corecte și
  complete a cerinței, calitatea interpretărilor din comentariul bloc și
  \textbf{răspunsul corect dat de dumeavostră la întrebările puse de către profesor.}}
\end{itemize}




\subsection{Cerințe}\label{cerinux21be}

\subsubsection{\texorpdfstring{Demo pentru generarea a \emph{k} liste
aleatoare sortate de dimensiuni diferite (având în total \emph{n}
elemente, unde \emph{n} și \emph{k} sunt date) și interclasarea a 2
liste
(5p)}{Demo pentru generarea a k liste aleatoare sortate de dimensiuni diferite (având în total n elemente, unde n și k sunt date) și interclasarea a 2 liste (5p)}}\label{demo-pentru-generarea-a-k-liste-aleatoare-sortate-de-dimensiuni-diferite-avuxe2nd-uxeen-total-n-elemente-unde-n-ux219i-k-sunt-date-ux219i-interclasarea-a-2-liste-5p}

\emph{Demo}: Corectitudinea algoritmului (\emph{generare și interclasare}) va trebui
demonstrată pe date de intrare de dimensiuni mici (ex: k=4, n=20). \scalebox{0.001}{Notă: pentru exemplul de test, lista trebuie afișată în ordine inversă, adică elementele acesteia trebuie afișate în ordine descrescătoare.}

\subsubsection{\texorpdfstring{Adaptare operațiilor de \emph{min-heap} pe
structura nouă și interclasarea a k liste
(3p)}{Adaptare operațiilor de min-heap pe structura nouă și interclasarea a k liste (3p)}}\label{adaptare-operaux21biilor-de-min-heap-pe-structura-nouux103-ux219i-interclasarea-a-k-liste-3p}

\emph{Demo}: Corectitudinea algoritmului (\emph{interclasare}) va trebui demonstrată
pe date de intrare de dimensiuni mici (ex: k=4, n=20).

\subsubsection{Evaluarea algoritmului în cazul mediu statistic
(2p)}\label{evaluarea-algoritmului-uxeen-cazul-mediu-statistic-2p}

Se cere analiza algoritmului în cazul \textbf{mediu statistic}. Pentru
cazul \textbf{mediu statistic} va trebui să repetați măsurătorile de
câteva ori. Din moment ce \textbf{k} și \textbf{n} pot varia, se va face
o analiză în felul următor:

\begin{itemize}
\item
  Se alege, pe rând, 3 valori constante pentru \emph{k} (\textbf{k1=5,
  k2=10, k3=100}); generează \emph{k} șiruri \textbf{aleatoare} sortate
  pentru fiecare valoare a lui \emph{k} astfel încât numărul elementelor
  din toate șirurile să varieze între \textbf{100 și 10000} cu un
  increment maxim de 400 (sugerăm 100); rulați algoritmul pentru toate
  valorile lui \emph{n} (pentru fiecare valoare a lui \emph{k});
  generați un grafic ce reprezintă \textbf{suma atribuirilor și a
  comparațiilor} făcute de acest algoritm pentru fiecare valoare a lui
  \emph{k} (în total sunt 3 curbe).
\item
  Se alege \emph{\textbf{n=10.000}}; valoarea lui \emph{k} va varia
  între \textbf{10 și 500} cu un increment de 10; generați \emph{k}
  șiruri \textbf{aleatoare} sortate pentru fiecare valoare a lui
  \emph{k} astfel încât numărul elementelor din toate șirurile să fie
  10000; testați algoritmul de interclasare pentru fiecare valoare a lui
  \emph{k} și generați un grafic care reprezintă \textbf{suma
  atribuirilor și a comparațiilor}.
\end{itemize}

\end{document}