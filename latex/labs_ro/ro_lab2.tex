\documentclass[../ro-fa-lab.tex]{subfiles}

\usepackage{hyperref}
\hypersetup{
    pdftitle={(RO) L2 - Construire heap si Heapsort},   % The title shown in the browser tab
    pdfauthor={},         % Your name or organization
    pdfsubject={},   % A brief description
    pdfkeywords={}
}

\begin{document}


\section{\texorpdfstring{\textbf{Tema Nr. 2: Analiza și Compararea a două metode de construire a
structurii de date Heap: ``De jos în sus'' (Bottom-up) vs. ``De sus în
jos'' (Top-down)}}{Tema Nr. 2: Analiza și Compararea a două metode de construire a
structurii de date Heap: ``De jos în sus'' (Bottom-up) vs. ``De sus în
jos'' (Top-down)}}\label{assign2}

\textbf{Timp de lucru:} 2 ore

\subsection{Implementare}\label{implementare}

Se cere implementarea \textbf{corectă} și \textbf{eficientă} a două
metode de construire a structurii de date Heap i.e., ``de jos în sus''
(\emph{bottom-up}) și ``de sus în jos'' (\emph{top-down}). De asemenea,
se cere implementarea algoritmului \emph{heapsort.}

\textsubscript{Informații utile și pseudo-cod găsiți în notițele de curs
sau în bibliografie}\cite{cormen}\textsubscript{:}

\begin{itemize}
\item
  ``\emph{De jos în sus}'': secțiunea 6.3
  (Building a heap)
\item
  \emph{Heapsort:} sectiunea 6.4 (The heapsort
  algorithm)
\item
  ``\emph{De sus în jos}'': secțiunea 6.5
  (Priority queues) și problema 6-1 (Building a heap using insertion)
\end{itemize}

\subsection{Cerințe minimale pentru
notare}\label{cerinux21be-minimale-pentru-notare}

\begin{itemize}
\item
  Interpretați graficul și notați observațiile personale în antetul
  fișierului \emph{main.cpp}, într-un comentariu bloc informativ.
\end{itemize}

\begin{itemize}
\item
  Pregătiți un exemplu pentru exemplificarea corectitudinii fiecărui
  algoritm implementat.
\item
  Nu preluăm teme care nu sunt indentate și care nu sunt organizate în
  funcții (de exemplu nu preluam teme unde tot codul este pus in main)
\end{itemize}

\begin{itemize}
\item
  \emph{\textbf{Punctajele din barem se dau pentru rezolvarea corectă și
  completă a cerinței, calitatea interpretărilor din comentariul bloc și
  răspunsul corect dat de voi la întrebările puse de către profesor.}}
\end{itemize}

\subsection{Cerințe}\label{cerinux21be}

\subsubsection{\texorpdfstring{Analiza comparativă a \emph{unuia} din
algoritmii de sortare din L1 (la alegere) în versiune \emph{iterativă}
vs \emph{recursivă}. Analiza se va efectua atât din \emph{perspectiva
\ul{numărului de operații}, cât și a \ul{timpului de rulare}}(2p)}{Analiza comparativă a unuia din algoritmii de sortare din L1 (la alegere) în versiune iterativă vs recursivă. Analiza se va efectua atât din perspectiva numărului de operații, cât și a timpului de rulare (2p)}}\label{analiza-comparativux103-a-unuia-din-algoritmii-de-sortare-din-l1-la-alegere-uxeen-versiune-iterativux103-vs-recursivux103.-analiza-se-va-efectua-atuxe2t-din-perspectiva-numux103rului-de-operaux21bii-cuxe2t-ux219i-a-timpului-de-rulare-2p}

Corectitudinea algoritmului va trebui demonstrată pe date de intrare de dimensiuni mici.

Pentru analiza comparativă a versiunii iterative vs recursive, alegeți
oricare din cei 3 algoritmi din laboratorul 1 (bubble sort, insertion
sau selection). Folosiți varianta iterativă pe care ați implementat-o și
predat-o în cadrul laboratorului (corectată, dacă este nevoie, în
funcție de feedback-ul pe care l-ați primit) și implementați același
algoritm de sortare în mod recursiv.

Trebuie să măsurați atât efortul total, cât și timpul de rulare al celor
două versiuni (iterativ și recursiv) =\textgreater{} două grafice,
fiecare comparând cele două versiuni de algoritm.

Pentru a măsura timpul de execuție, puteți folosi Profiler, similar cu
exemplul de mai jos.
\\

\emph{profiler.startTimer("your\_function", current\_size);}

\emph{for(int test=0; test\textless nr\_tests; ++test) \{}

\quad\emph{your\_function(array, current\_size);}

\emph{\}}

\emph{profiler.stopTimer("your\_function", current\_size);}
\\


Numărul de teste (\emph{nr\_tests} din exemplul de mai sus) trebuie ales
în funcție de procesor și modul de compilare. Sugerăm valori mai mari,
precum 100 sau 1000.

În momentul în care măsurați timpul de execuție, asigurați-vă că opriți
orice alte procese care nu sunt necesare.

\emph{Observație}. Pentru a evalua cât mai corect timpul, Profiler nu va face numărătoarea operațiilor. Astfel, evaluarea de timp și de operații trebuie făcute separat.

\subsubsection{Implementarea metodei bottom-up de construire a unui heap
(2p)}\label{implementarea-metodei-bottom-up-de-construire-a-unui-heap-2p}

Corectitudinea algoritmului va trebui demonstrată pe date de intrare de
dimensiuni mici.

\subsubsection{Implementarea metodei top-down de construire a unui heap
(2p)}\label{implementarea-metodei-top-down-de-construire-a-unui-heap-2p}

Corectitudinea algoritmului va trebui demonstrată pe date de intrare de
dimensiuni mici.

\subsubsection{Analiza comparativă a celor două metode de construire în
cazul mediu statistic
(2p)}\label{analiza-comparativux103-a-celor-douux103-metode-de-construire-uxeen-cazul-mediu-statistic-2p}

\textbf{!} Înainte de a începe să lucrați pe partea de evaluare,
asigurați-vă că aveți un algoritm \emph{corect}!

Se cere compararea celor două metode de construcție a structurii heap
în cazul mediu statistic. Pentru cazul mediu statistic, va trebui să
repetați măsurătorile de m ori (m=5) și să raportați valoarea lor medie;
de asemenea, pentru cazul mediu statistic, asigurați-vă că folosiți
aceleași date de intrare pentru cele două metode.

Pașii de analiză:

- variați dimensiunea șirului de intrare (\emph{n}) între
{[}100\ldots10000{]}, cu un increment de maxim 500 (sugerăm 100).

- pentru fiecare dimensiune (\emph{n}), generați date de intrare
adecvate metodei de construcție; rulați metoda numărând operațiile
elementare (atribuiri și comparații - pot fi numărate împreună pentru
această temă).

\textbf{!} Doar atribuirile și comparațiile care se fac pe datele de
intrare și pe datele auxiliare corespunzătoare se iau în considerare.

Generați un grafic ce compară cele două metode de construcție în cazul
mediu statistic pe baza numărului de operații obținut la pasul anterior.
Dacă o curbă nu poate fi vizualizată corect din cauza că celelalte curbe
au o rată mai mare de creștere, atunci plasați noua curbă și pe un alt
grafic. Denumiți adecvat graficele și curbele.

\subsubsection{Analiza comparativă a metodelor de construcție în cazul
defavorabil
(1p)}\label{analiza-comparativux103-a-metodelor-de-construcux21bie-uxeen-cazul-defavorabil-1p}

\subsubsection{Implementarea și exemplificarea corectitudinii algoritmului
heapsort
(1p)}\label{implementarea-ux219i-exemplificarea-corectitudinii-algoritmului-heapsort-1p}

Corectitudinea algoritmului va trebui demonstrată pe date de intrare de
dimensiuni mici.


\printbibliography

\end{document}