\documentclass[../en-fa-lab.tex]{subfiles}

\usepackage{hyperref}
\hypersetup{
    pdftitle={(EN) L1 - Direct sorting methods},   % The title shown in the browser tab
    pdfauthor={},         % Your name or organization
    pdfsubject={},   % A brief description
    pdfkeywords={}
}

\begin{document}


\section{\texorpdfstring{\textbf{Assignment No. 1: Analysis \& Comparison of Direct Sorting
Methods}}{Assignment No. 1: Analysis \& Comparison of Direct Sorting
Methods}}\label{assign1}

\textbf{Allocated time:} 2 hours

\subsection{Implementation}\label{implementation}

You are required to implement \textbf{correctly} and
\textbf{efficiently} 3 direct sorting methods (\emph{Bubble Sort},
\emph{Insertion Sort} -- using either linear or binary insertion and
\emph{Selection Sort})

\begin{itemize}
\item
  Input: sequence of numbers \(< a_{1},\ a_{2},\ \ldots,\ a_{n} >\)
\item
  Output: an ordered permutation of the input sequence
  \({< a}_{1}' \leq a_{2}' \leq \ldots \leq a_{n}' >\)
\end{itemize}

You may find any necessary information and pseudo-code in the
\textbf{Seminar no. 1 notes} (Insertion Sort is also presented in the
\textbf{book\cite{cormen}-- Section 2.1}). Make sure that you
implement the efficient version for each of the required sorting methods
(if more than one version has been provided to you).

\textbf{Minimal requirements for grading}

\begin{itemize}
\item
  Interpret the chart and write your observations in the header (block
  comments) section at the beginning of your \emph{main.cpp} file.
\end{itemize}

\begin{itemize}
\item
  Prepare a demo for each algorithm implemented.
\item
  We do not accept assignments without code indentation and with code
  not organized in functions (for example where the entire code is in
  the main function).
\item
  \emph{\textbf{The points from the requirements correspond to a correct
  and complete solution, quality of interpretation from the block
  comment and the correct answer to the questions from the teacher.}}
\end{itemize}

\subsection{Requirements}\label{requirements}

\subsubsection{Implementation of direct sorting method
(5.5p)}\label{implementation-of-direct-sorting-method-5.5p}

\begin{itemize}
\item
  Bubble sort (1.5p)
\item
  Insertion sort (2p)
\item
  Selection sort (2p)
\end{itemize}

You will have to prove your algorithm(s) work, so you should also
prepare a demo on a small-sized input (which may be hard-coded in your
\emph{main} function).

\subsubsection{Evaluate algorithms for the average case (1.5p -- 0.5p for
each
algorithm)}\label{evaluate-algorithms-for-the-average-case-1.5p-0.5p-for-each-algorithm}

\textbf{!} Before you start to work on the algorithms evaluation code,
make sure you have a correct algorithm!

You are required to compare the three sorting algorithms, in the
\textbf{best}, \textbf{average} and \textbf{worst} cases. Remember that
for the \textbf{average} case you have to repeat the measurements
\textbf{m} times (m=5 should suffice) and report their average; also,
for the \textbf{average} case, make sure you always use the
\textbf{same} input sequence for all three sorting methods. To make the
comparison fair, make sure you know how to generate the
\textbf{best/worst} case input sequences for all three methods.

This is how the analysis should be performed for a sorting method, in
any of the three cases (\textbf{best}, \textbf{average} and
\textbf{worst}):

- vary the dimension of the input array (\emph{n}) between
{[}100\ldots10000{]}, with an increment of maximum 500 (we suggest 100).

- for each dimension, generate the appropriate input sequence for the
sorting method; run the sorting method counting the operations (i.e.,
number of assignments, number of comparisons and their sum).

\begin{quote}
\textbf{!} Only the assignments (=) and comparisons (\textless, ==,
\textgreater, !=) which are performed on the input structure and its
corresponding auxiliary variables matter.
\end{quote}

\subsubsection{Evaluate algorithm for best and worst case (3p - 0.5p for
each case of each
algorithm)}\label{evaluate-algorithm-for-best-and-worst-case-3p---0.5p-for-each-case-of-each-algorithm}

For each analysis case (\textbf{best}, \textbf{average,} and
\textbf{worst}), generate charts which compare the three methods; use
different charts for the number of comparisons, number of assignments
and total number of operations. If one of the curves cannot be
visualized correctly because the others have a larger growth rate (e.g.,
a linear function might seem constant when placed on the same chart with
a quadratic function), place that curve on a separate chart as well.
Name your charts and the curves on each chart appropriately.

\subsubsection{Bonus: Binary insertion sort
(0.5p)}\label{bonus-binary-insertion-sort-0.5p}

You will have to prove your algorithm(s) work, so you should also
prepare a demo on a small-sized input (which may be hard-coded in your
main function).

You will have to compare binary insertion sort against all other sorting
methods (bubble, insertion, selection) on all cases (best, average,
worst).

\printbibliography

\end{document}