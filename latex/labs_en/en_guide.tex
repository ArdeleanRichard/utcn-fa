\documentclass[../en-fa-lab.tex]{subfiles}


\usepackage{hyperref}
\hypersetup{
    pdftitle={Laboratory Guide},   % The title shown in the browser tab
    pdfauthor={},         % Your name or organization
    pdfsubject={},   % A brief description
    pdfkeywords={}
}


\begin{document}

\section{\texorpdfstring{\textbf{Laboratory Guide}}{Laboratory Guide}}\label{laboratory-guide}
\subsection{\texorpdfstring{\textbf{About the
laboratory}}{About the laboratory}}\label{about-the-laboratory}

This semester, you will implement a series of algorithms and you will
analyze the \emph{correctness} and the \emph{efficiency} of your own
implementations. The implementations will have a starting point the
pseudocode that is provided at the \textbf{course} and \textbf{seminary}
sessions. You can write your code in C or C++ or another language that
you are familiar with as long as you implement all the data structures
that are required.

\subsection{\texorpdfstring{\textbf{Laboratory
Format}}{Laboratory Format}}\label{laboratory-format}

\subsubsection{\texorpdfstring{\textbf{Rules}}{Rules}}\label{rules}

\begin{itemize}
\item
  \emph{Attendance} is \textbf{mandatory}
\item
  \emph{One} absence can be recovered (the corresponding assignment
  \ul{MUST} be \emph{delivered} and \emph{verified} the following week)
\item
  A second absence can be recovered in the special laboratory session at
  the end of the semester (\ul{surcharge}) by solving additional
  assignments (solving the corresponding assignment of the missed
  session will NOT be considered)
\item
  If you miss more than 2 sessions, \textbf{you will not be able to
  attend the exam in the regular exam session}
\item
  \textbf{IN EXCEPTIONAL CASES}, you can attend another laboratory
  sessions in the same week with another group \emph{if} you announce
  your laboratory assistant on email and you receive his agreement; the
  assignment will be delivered to and verified by the laboratory
  assistant of the sessions you are signed up for (the assignment has to
  be uploaded on time and will be presented the following week at the
  laboratory session)
\end{itemize}

\subsubsection{\texorpdfstring{\textbf{Grading}}{Grading}}\label{grading}

\begin{itemize}
\item
  The laboratory grade is equal to \textbf{30\%} of the final grade
\item
  The laboratory grade is composed of two parts: \textbf{assignments grade} - 2/3 of lab grade (that is \textbf{20\%} of final grade) and \textbf{colloquy/lab exam} - 1/3 of lab grade (that is \textbf{10\%} of final grade) 
\item Assignments:
\begin{itemize}
    \item
      Each assignment has the same weight in the calculation of the
      assignments grade - unweighted arithmetic mean formula is used
    \item
      Each assignment has 4 grading thresholds for the following grades: 5,
      7, 9 and 10 (the requirements for each grade can be viewed in the
      laboratory assignment documentation)
    % \item
    %   Passing  to the exam is conditioned by a grade (determined by the
    %   formula mentioned above) of \textbf{at least 5 in the laboratory}
\end{itemize}
\item Colloquy:
\begin{itemize}
    \item The colloquy consists of a closed book lab test in the last week of the semester (W14)
    \item The colloquy must be held by each student on the computers available in the laboratory room (no personal laptops / computers) 
\end{itemize}

\item In order to pass the laboratory and be granted participation in the exam you need \textbf{both}:
\begin{itemize}
    \item Assignments grade $\geq 5$
    \item Colloquy grade $\geq\ 5$
\end{itemize}

\end{itemize}

%\subsubsection{\texorpdfstring{\textbf{Assignments}}{Assignments}}\label{assignments}

\subsubsection{\texorpdfstring{\textbf{Assignments Delivery}}{Assignments Delivery}}\label{delivery}

\begin{itemize}
\item
  At the assessment discussion you will present: \textbf{source code,
  charts and runnable demo code}
\item
  \textbf{Source code} (program.cpp) and the \textbf{charts} \textbf{MUST}
  be uploaded to Moodle, in the archive (program.zip), \textbf{before the lab session}
\item
  \emph{Unindented} code will \textbf{not be evaluated}
\item
  If you cannot explain the used algorithms, the assignment will
  \textbf{not be evaluated}
\item
  Each source file must contain at the beginning a comment of the
  following format:
\end{itemize}

/**

* @author John Smith

* @group 30xyz

*

* Assignment requirements, ex: Compare the sorting methods X and Y

*

* Personal interpretation of the (time and space) complexity of the
average, best and worst testing Interpretarea

* cases. For example: ``Method X has a time complexity of Y in case Z
because \ldots''

*/

\subsubsection{\texorpdfstring{\textbf{Algorithm complexity
evaluation}}{Algorithm complexity evaluation}}\label{algorithm-complexity-evaluation}

\begin{itemize}
\item
  For the average case, you need to repeat the measurements \emph{at
  least 5 times}
\item
  Measure the number of operations made by the algorithm (the
  assignments and the comparisons of the input data or the auxiliary
  variables that contain input data)
\item
  Vary the dimensions of the input data consistent with the requirements
  of each assignment
\item
  Apply the same input data on each algorithm when making comparative
  evaluations (for the average case)
\item
  Generate evaluation charts (either in \textbf{Excel} or by using
  \textbf{Profiler})
\item
  Analyze the charts and add your personal observations at the beginning
  of the source code using the above mentioned format
\end{itemize}

\subsubsection{\texorpdfstring{\textbf{Delivery
deadline}}{Delivery deadline}}\label{delivery-deadline}

Assignments can be delivered:

\begin{itemize}
\item
  During the laboratory sessions that they are presented in. At the end
  of the laboratory you \ul{MUST} upload to Moodle a draft version of
  the current assignment (containing what you managed to implement
  during the session). A lack of a submission at the end of the session
  (or a lack of relevant code) will receive of penalty of up to 2 points
  of the grade of that assignment
\item
  \textbf{Extension\_1} (E1): at the beginning of the next laboratory
  session
\item
  \textbf{Extension\_2} (E2): \ul{specific} assignments can be delivered
  at the beginning of the second laboratory after the assignment was
  presented (with a penalty of -2).
\item
  Starting with the third laboratory after the assignment was presented,
  the assignment cannot be delivered anymore (will be graded as 0)
\end{itemize}

You can find a planning of the assignment and their extensions on Moodle
(file name: \ul{Planificare Saptamanala (bachelor)} )

Each assignment \ul{MUST} be uploaded to Moodle before the beginning of
the laboratory session during which it will be delivered.

\subsubsection{\texorpdfstring{\textbf{Attempted
Fraud}}{Attempted Fraud}}\label{attempted-fraud}

For the first uncovered attempt at fraud (copying someone else's code or using code generated by AI tools)
you will receive a penalty of 10 points of the total accumulated points
(until that moment). For a subsequent attempt, you will have to \textbf{retake the course next year}.

\subsection{\texorpdfstring{\textbf{Laboratory session
transfer}}{Laboratory session transfer}}\label{laboratory-session-transfer}

If you wish to participate at laboratory sessions with another assistant
you \ul{must} follow these rules:

\begin{itemize}
\item
  Student S1 from group G1 can transfer to group G2 if and only if a
  student S2 from group G2 can be found that is willing to participate
  with G1 at the assigned laboratory session hours of G1.
\end{itemize}

To formalize the transfer you must send an email in which you mention
with whom are you making the transfer/exchange:

\begin{itemize}
\item
  An email from S1 to both laboratory assistants
\item
  An email from S2 to both laboratory assistants
\end{itemize}

The laboratory session transfer deadline is the \textbf{end of the
second week} of the semester.

\subsection{\texorpdfstring{\textbf{Suggested
literature}}{Suggested literature}}\label{suggested-literature}

\begin{itemize}
\item
    \href{https://drive.google.com/drive/folders/12NX3D8DLrPuU-9ffI9MjnoKUNjjftWRg?usp=sharing}{Cormen, T. H. et al (2009). Introduction to algorithms. MIT press}
  
\item
    \href{https://drive.google.com/drive/folders/12NX3D8DLrPuU-9ffI9MjnoKUNjjftWRg?usp=sharing}{J. Kleinberg, E. Tardos (2005). Algorithm Design. Addison Wesley}

\item
  C/C++ Tutorials

  \begin{itemize}
  \item
    \href{http://www.cprogramming.com/begin.html}{\ul{http://www.cprogramming.com/begin.html}}
  \item
    \href{http://www.learn-c.org}{\ul{http://www.learn-c.org}}
  \item
    Accelerated C++: Practical Programming by Example
  \end{itemize}
\item
  Coding Styles

  \begin{itemize}
  \item
    \href{http://users.ece.cmu.edu/~eno/coding/CCodingStandard.html}{\ul{http://users.ece.cmu.edu/\textasciitilde eno/coding/CCodingStandard.html}}
  \item
    \href{http://www.cs.swarthmore.edu/~newhall/unixhelp/c_codestyle.html}{\ul{http://www.cs.swarthmore.edu/\textasciitilde newhall/unixhelp/c\_codestyle.html}}
  \item
    \href{http://google-styleguide.googlecode.com/svn/trunk/cppguide.xml}{\ul{http://google-styleguide.googlecode.com/svn/trunk/cppguide.xml}}
  \end{itemize}
\end{itemize}


\subsection{\texorpdfstring{\textbf{Profiler}}{Profiler}}\label{profiler}
The library that will be utilized for the generation of plots, each student must go through the given example and tutorial.
\\\\
The most recent version can be found here: 

https://github.com/cypryoprisa/utcn-fa-profiler
\\\\
Profiler Tutorial:

- \href{https://youtu.be/iE4FFwdncDk}{Part 1}

- \href{https://youtu.be/vMhwBXEelSA}{Part 2}

- \href{https://youtu.be/ZOztj7aPWs4}{Part 3}


\end{document}