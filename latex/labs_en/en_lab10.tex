\documentclass[../en-fa-lab.tex]{subfiles}

\usepackage{hyperref}

\hypersetup{
    pdftitle={(EN) L10 - DFS},   % The title shown in the browser tab
    pdfauthor={},         % Your name or organization
    pdfsubject={},   % A brief description
    pdfkeywords={}
}


\begin{document}

\section{\texorpdfstring{\textbf{Assignment 10: Depth-first search (DFS)}}{Assignment 10: Depth-first search (DFS)}}\label{assign10}
\textbf{}

\textbf{Allocated time: 2 hours}

\subsection{Implementation}\label{implementation}

You are required to correctly and efficiently implement the depth first
search algorithm (DFS) (\emph{Chapter 22.3 from \cite{cormen}}). For
graph representation, you should use adjacency lists. You also have to:

\begin{itemize}
\item
  Implement the Tarjan algorithm for detecting strongly connected
  components
\item
  Implement topological sorting (\emph{Chapter 22.4 from \cite{cormen}})
\end{itemize}

\subsection{Requirements}\label{requirements}

\subsubsection{DFS (5p)}\label{dfs-5p}

Exemplify the correctness of your algorithm/implementation by running it
on a smaller graph:

\begin{itemize}
\item
  print the initial graph (the adjacency lists)
\item
  print the tree resulted from DFS
\end{itemize}

\subsubsection{Topological sort (1p)}\label{topological-sort-1p}

Exemplify the correctness of your algorithm/implementation by running it
on a smaller graph:

\begin{itemize}
\item
  print the initial graph (the adjacency lists)
\item
  print a list of nodes sorted topologically (should this list be
  nonempty/if it is why so?)
\end{itemize}

\subsubsection{Tarjan (2p)}\label{tarjan-2p}

Exemplify the correctness of your algorithm/implementation by running it
on a smaller graph:

\begin{itemize}
\item
  print the initial graph (the adjacency lists)
\item
  print all strongly connected components of the graph
\end{itemize}

\subsubsection{Analysis of the DFS performance
(2p)}\label{analysis-of-the-dfs-performance-2p}

\textbf{!} Before you start to work on the algorithm evaluation code,
make sure you have a correct algorithm!

Since, for a graph, both \textbar V\textbar{} and \textbar E\textbar{}
may vary, and the running time of DFS depends on both, we will make each
analysis in turn:

\begin{enumerate}
\def\labelenumi{\arabic{enumi}.}
\item
  Set \textbar V\textbar{} = 100 and vary \textbar E\textbar{} between
  1000 and 4500, using a 100 increment. Generate the input graphs
  randomly -- make sure you don't generate the same edge twice for the
  same graph. Run the DFS algorithm for each graph and count the number
of operations performed; generate the corresponding chart (i.e., the variation of the number of operations with \textbar E\textbar).
\item
  Set \textbar E\textbar{} = 4500 and vary \textbar V\textbar{} between
  100 and 200, using an increment equal to 10. Repeat the procedure
  above to generate the chart which gives the variation of the number of
  operations with \textbar V\textbar.
\end{enumerate}


\printbibliography

\end{document}