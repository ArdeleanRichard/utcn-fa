\documentclass[../en-fa-lab.tex]{subfiles}
\usepackage{booktabs}
\usepackage{tabularx}
\usepackage{caption}

\usepackage{hyperref}

\hypersetup{
    pdftitle={(EN) L5 - Hash tables},   % The title shown in the browser tab
    pdfauthor={},         % Your name or organization
    pdfsubject={},   % A brief description
    pdfkeywords={}
}

\begin{document}

\section{\texorpdfstring{\textbf{Assignment No. 5: Search Operation in Hash Tables}}{Assignment No. 5: Search Operation in Hash Tables}}\label{assign5}

\textbf{Open Addressing with Quadratic Probing}

\noindent\textbf{Allocated time:} 2 hours

\subsection{Implementation}\label{implementation}

You are required to implement \textbf{correctly} and
\textbf{efficiently} the \emph{insert} and \emph{search} operations in a
hash table using \emph{open addressing} and \emph{quadratic probing}.

You may find relevant information and pseudo-code in your course notes,
or in the book (\cite{cormen}), in section \emph{11.4 Open addressing}.

The notions of closed/open specify whether you are compelled to use a
certain position or a data structure.

\subsubsection{Hashing (refers to the hash
table)}\label{hashing-refers-to-the-hash-table}

\begin{itemize}
\item
  Open Hashing: free to leave the hash table to hold more elements at
    a certain index (e.g. chaining)
\item
  Closed Hashing: not more than one element can be stored at a certain
    index (e.g. linear/quadratic probing)
\end{itemize}

\subsubsection{Addressing (refers to the final position of the element with
respect to its initial
position)}\label{addressing-refers-to-the-final-position-of-the-element-with-respect-to-its-initial-position}

\begin{itemize}
\item
  Open Addressing: the final address is not completely determined by the
    hash code, it also depends on the elements which are already in the
    hash table (e.g. linear/quadratic probing)
\item
  Closed Addressing: the final address is always the one initially
    calculated (there is no probing, e.g. chaining)
\end{itemize}

\textsubscript{\hfill\break
}

For the purpose of this assignment, the hash table will
not contain integers, but a custom data structure defined as follows:



\begin{verbatim}
typedef struct {
    int  id;
    char name[30];
} Entry;
\end{verbatim}


where the \emph{position} of each Entry in the Hash Table will be calculated
by applying the required hash function on the \emph{id} member of the
struct. The \emph{name} member of the struct will be used only to
exemplify the correctness of the search and delete operations and is not
needed when evaluating the performance (i.e., the \emph{name} member
will be printed to the console if the search operation finds the
\emph{id}, otherwise print ``not found'').






\subsection{Minimal requirements for grading}\label{minimal-requirements-for-grading}

The lack of any of the minimum requirements (even partially) may result in a lower grade through penalties or refusal to accept the assignment resulting in a grade of 0.

\begin{itemize}
\item
  \emph{Demo}: Prepare a demonstration of correctness for each algorithm implemented. The correctness of each algorithm is demonstrated through a simple example (maximum 10 values).
\item
  The charts created must be easy to evaluate as in grouped and added through the Profiler functions as specified by the assignment requirements. The assignment will not be evaluated if it contains a plethora of ungrouped charts. For example, the comparative analysis implies the grouping of the compared algorithms.
\item
  Interpret the chart and write your observations in the header (block
  comments) section at the beginning of your \emph{main.cpp} file.
\item
  We do not accept assignments without code indentation and with code
  not organized in functions (for example where the entire code is in
  the main function).
\item
  \emph{The points from the requirements correspond to a correct
  and complete solution, quality of interpretation from the block
  comment and \textbf{the correct answer to the questions from the teacher.}}
\end{itemize}






\subsection{Requirements}\label{requirements}

\subsubsection{Implementation of the insert and search operations using the
required data structure
(5p)}\label{implementation-of-the-insert-and-search-operations-using-the-required-data-structure-5p}

\emph{Demo}: You will have to prove your algorithm(s) work on a small-sized input
\emph{(ex. 10)}.

\subsubsection{Evaluate the search operation for a single fill factor 95\%
(2p)}\label{evaluate-the-search-operation-for-a-single-fill-factor-95-2p}

You are required to evaluate the \emph{search} operation for hash tables
using open addressing and quadratic probing, in the \textbf{average
case} (remember to perform 5 runs for this). You will do this in the
following manner:

\begin{enumerate}
\def\labelenumi{\arabic{enumi}.}
\item
  Select \emph{N}, the size of your hash table, as a prime number around
  10000 (e.g., 9973, or 10007);
\item
  For each of several values for the filling factor \textbf{\emph{$\alpha$} =0.95} do:

  \begin{enumerate}
  \def\labelenumii{\alph{enumii}.}
  \item
    Insert n random elements, such that you reach the required value for
    \emph{$\alpha$} (\emph{$\alpha$ = n/N})
  \item
    Search, in each case, \emph{m} random elements ($\emph{m}=3000$), 
    such that approximately half of the
    searched elements will be \emph{found} in the table, and the rest
    will \emph{not} be \emph{found} (in the table). \emph{Make sure that
    you sample uniformly the elements in the found category, i.e., you
    should search elements which have been inserted at different moments
    with equal probability (there are several ways in which you could
    ensure this -- it is up to you to figure this out)}
  \item
    Count the operations performed by the search procedure (i.e., the
    number of cells accessed during the search)
  \item
    Pay attention to the values that you search for, they should be in
    random order of introduction. \emph{If you look for the first 1500
    values introduced in the table, implicitly the average found effort
    will be 1.}
  \end{enumerate}
\item
  Output a table in the following form:
\end{enumerate}

\begin{table}[ht]
  \centering
  \caption{Effort measurements at various filling factors}
  \label{tbl:effort-vs-filling}
  \begin{tabularx}{\textwidth}{%
      >{\centering\arraybackslash}X
      >{\centering\arraybackslash}X
      >{\centering\arraybackslash}X
      >{\centering\arraybackslash}X
      >{\centering\arraybackslash}X
    }
    \toprule
    \textbf{Filling factor}
      & \textbf{Avg. Effort \emph{(found)}}
      & \textbf{Max Effort \emph{(found)}}
      & \textbf{Avg. Effort \emph{(not-found)}}
      & \textbf{Max Effort \emph{(not-found)}} \\
    \midrule
    0.95 & … & … & … & … \\
    % add more rows here as needed
    \bottomrule
  \end{tabularx}
\end{table}

\emph{Avg. Effort = total\_effort / no\_elements}

\emph{Max. Effort = maximum number of accesses performed by one search
operation}

\subsubsection{Complete evaluation for all fill factors
(2p)}\label{complete-evaluation-for-all-fill-factors-2p}

Respecting the requirements of point 2 with \emph{$\alpha$} $\in$ \{0.8, 0.85, 0.9, 0.95, 0.99\}, output a table in the following form:

\begin{table}[ht]
  \centering
  \caption{Effort measurements at various filling factors}
  \label{tbl:effort-vs-filling-extended}
  \begin{tabularx}{\textwidth}{%
      >{\centering\arraybackslash}X
      >{\centering\arraybackslash}X
      >{\centering\arraybackslash}X
      >{\centering\arraybackslash}X
      >{\centering\arraybackslash}X
    }
    \toprule
    \textbf{Filling factor}
      & \textbf{Avg. Effort \emph{(found)}}
      & \textbf{Max Effort \emph{(found)}}
      & \textbf{Avg. Effort \emph{(not-found)}}
      & \textbf{Max Effort \emph{(not-found)}} \\
    \midrule
    0.80  &     &     &     &     \\
    0.85  &     &     &     &     \\
    …     & …   &     & …   &     \\
    \bottomrule
  \end{tabularx}
\end{table}

\subsubsection{\texorpdfstring{Implement delete operation in a hash table,
\emph{demo (size 10)} and evaluation of the search operation after
deletion of some elements
(1p)}{Implement delete operation in a hash table and evaluation of the search operation after deletion of some elements (1p)}}\label{implement-delete-operation-in-a-hash-table-demo-size-10-and-evaluation-of-the-search-operation-after-deletion-of-some-elements-1p}

\emph{Demo}: You will have to prove your algorithm(s) work on a small-sized input
\emph{(ex. 10)}.

For the evaluation of the search operation after deletion, fill the hash
table until a fill factor of 0.99. Delete elements from the table until
you get a filling factor of 0.8 and afterwards search m random elements
(m \textasciitilde{} 3000) such that approximately half of the searched
elements will be \emph{found} in the table, and the rest will \emph{not be found} (in the table)\emph{.} Count the operations performed by the \emph{search} and add it in the previous table.


\printbibliography

\end{document}